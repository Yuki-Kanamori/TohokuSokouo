\documentclass[11pt]{article} %フォントサイズ指定
\usepackage{times}
\usepackage{plext}
\usepackage{cite}
\usepackage{setspace}
\usepackage{float}
\usepackage{lineno}
\usepackage{color}
\usepackage[dvipdfmx]{graphicx}
\usepackage{array, booktabs}
\usepackage[top=25truemm,bottom=25truemm,left=25truemm,right=25truemm]{geometry} %マージン指定
\usepackage{threeparttable}
\usepackage{ulem}
\usepackage{amsmath}
\renewcommand{\figurename}{Fig.}
\renewcommand{\tablename}{Table}

\makeatletter
\def\mojiparline#1{
\newcounter{mpl}
\setcounter{mpl}{#1}
\@tempdima=\linewidth
\advance\@tempdima by-\value{mpl}zw
\addtocounter{mpl}{-1}
\divide\@tempdima by \value{mpl}
\advance\kanjiskip by\@tempdima
\advance\parindent by\@tempdima
}
\makeatother
\def\linesparpage#1{
\baselineskip=\textheight
\divide\baselineskip by #1
}

\begin{document}
%\setcounter{page}{1}
%\renewcommand\citeleft{(} 
%\renewcommand\citeright{)}

\mojiparline{40} %X:一行あたり文字数の指定
\linesparpage{45} %Y:1ページあたり行数の指定


\flushleft{\textbf{\Large{2020年底魚類現存量調査結果}}}\\
\flushleft{金森由妃$^{\ast}$, 成松庸二, 鈴木勇人, 森川英祐, 時岡 駿, 三澤 遼, 永尾次郎(水産資源研究所)}\\
\ \\
$^{\ast}$ Email: kana.yuki@fra.affrc.go.jp\\

\begin{linenumbers}
\section{はじめに}
我が国が1996年に批准した国連海洋法条約は,批准国に領海内の水産資源を適切に管理することを義務づけている.このため(国研)水産研究・教育機構 東北区水産研究所では,1995年から東北地方太平洋岸沖(東北海域)において毎年秋季に主要底魚類の資源量調査を実施し,その資源状態を調査している.本報告は2020年に行なった調査結果から主要魚種の現存量,分布および体長組成を推定し,過去の調査結果と比較することで東北海域における主要底魚類の資源状況を把握することを目的とした.

\section{材料と方法}
2020年10月1日から11月13日に,青森県尻屋崎沖(北緯$\textrm{41.2}^\circ$)から茨城県日立沖(北緯$\textrm{36.5}^\circ$)までの海域において,調査船若鷹丸(水産研究・教育機構所属,692トン)による着底トロール調査を実施した.調査海域には等深線を横切る8本の調査ライン(A-Hライン)と8層の水深帯(150-900m帯)を設定し,この内に調査点を設置した(図1).なおC, Dライン(宮古,釜石沖)の浅海域については,地形が曳網に適さないこと,定置網や刺網の漁場となっていることから210m以深を調査点とした.またズワイガニの現存量推定精度の向上を目的として,Dライン以南の各ライン間(DE-GHライン)に水深250-510m帯の調査点を設定した.

\ \ \ \ \ \ \ \ \ \ 
調査には袖網長13.0m,身網長26.1m,網口幅5.4m,コッドエンド長5.0mのトロール網を使用した.またこの網のコッドエンドは内網,外網,擦れ防止網の三重構造で,目合はそれぞれ50mm,8mm,60mmである.

\ \ \ \ \ \ \ \ \ \ 
昼夜で鉛直分布が変化する魚種の遭遇率や採集効率を一定にするため,調査は日の出から日没までの間に行なった.1調査点あたりの曳網時間は原則30分とし,漁業者への影響や破網,漁獲物の大量入網などの可能性がある場合には曳網時間を短縮した.曳網面積等の算出のため,網の離着底時,ワープセット時,揚網開始時には緯度経度,水深,ワープ長およびオッターボード間隔を計測した.曳網距離は網の着底から離底までとし,北川・服部(1998)の方法により計算した.網の袖先間隔はオッター間隔センサー(Marport社製,トロールフィッシュシステム)で計測したオッターボード間隔より推定した。これらの数値を用いて各調査点における曳網面積を推定した.

\ \ \ \ \ \ \ \ \ \ 
漁獲物は船上で魚種別に分類し,尾数と重量を測定した.また,スケトウダラ,マダラ,イトヒキダラ,キチジ,ズワイガニ,スルメイカ,ベニズワイ,アカガレイ,サメガレイ,およびババガレイについては体サイズ(魚類は全長TLおよび標準体長SL,スルメイカは外套長ML,カニ類は甲幅CW)を計測した.体長からスケトウダラは0歳魚と1歳魚以上に,マダラは0歳魚,1歳魚および2歳魚以上に区別した.ズワイガニおよびベニズワイは腹節の形状から雌雄を区別した.

\ \ \ \ \ \ \ \ \ \ 
漁獲個体数と曳網面積から,各調査点における分布密度を魚種別に推定した.なお, DE,EF,FG,GHラインはそれぞれE,F,G,Hラインに統合した.A~Dラインを北部海域,E~Hラインを南部海域とし,面積-密度法を用いて南北海域別に現存量・現存尾数を推定した。また現存尾数で引き延ばした体長組成を求め,過去の調査結果と比較した.なお,全魚種において採集効率は1を仮定した.


\section{結果と考察}
\subsection{スケトウダラ0歳魚}
2020年のスケトウダラ0歳魚は東北海域全体で分布密度が低く,2015年や2017年に分布密度が高い調査点が見られた北部の水深250~350m帯でも密度の高い調査点は見られなかった(図2).

\ \ \ \ \ \ \ \ \ \ 
スケトウダラ0歳魚の現存量,現存尾数は年変動が大きく,近年は2013年以降減少を続けていたが,2017年は現存尾数,現存量ともに2016年を大きく上回った(図3).2018年の現存尾数は海域全体では前年比0.2倍の14.4百万尾であった.海域別にみると北部海域は前年比0.1倍の6.7百万尾,南部海域は前年比0.6倍の7.8百万尾となり,北部海域における減少が顕著であった.2018年の現存量は海域全体では前年比0.3倍の0.4千トンであった(図3).海域別にみると北部海域は前年比0.1倍の0.1千トン,南部海域は前年比0.5倍の0.3千トンとなり,現存尾数と同様に北部海域における減少が顕著であった.

\ \ \ \ \ \ \ \ \ \ 
2018年の体長組成を見ると最頻値は北部海域で10cm, 南部海域で15cmとなり,2017年よりもやや小型の個体が中心となった(図4).

\ \ \ \ \ \ \ \ \ \ 
\textcolor{red}{
東北海域のスケトウダラは,北海道太平洋の資源と同一系群とされており,主産卵場である噴火湾周辺で産出された卵稚仔は,その一部が東北海域の北部まで移送されると考えられている(大迫ほか1986,橋本・石戸1987).また,東北海域での加入は,親潮第一分枝の流入強度が強い年に増加することが示唆されている(Hattori et al., 2006).気象庁発表の春季(3-5月)の親潮平均南限位置
%(https://www.data.jma.go.jp/gmd/kaiyou/data/shindan/b_2/oyashio_exp/oyashio_exp.html, 2019年4月2日閲覧)
と現存尾数の関係をみると,春季に親潮が南下するほど現存尾数が多い傾向があるものの($r^{2}$ = 0.237, $p < 0.05$),年によるばらつきが大きい(図5).東北海域におけるスケトウダラ0歳魚の加入と海洋環境との関係については,今後さらなる検討が必要と考えられる.
}

\subsection{スケトウダラ1歳魚以上}
スケトウダラ1歳魚以上は例年A~Cラインの水深250~350m帯で高密度の分布が確認されることが多い.2016年,2017年では10千尾/km2以上の分布が見られたのは2016年におけるAラインの水深250m帯のみと分布密度は低かったが,2018年ではAラインとBラインの水深250m帯および水深350m帯において10千尾/km2以上の高密度分布が見られた(図6).

\ \ \ \ \ \ \ \ \ \ 
2018年の現存尾数は,海域全体では前年比3.1倍の26.5百万尾となった(図7).海域別にみると北部海域は前年比3.7倍の23.9百万尾,南部海域は前年比1.2倍の2.6百万尾となり,北部海域における増加が顕著であった.現存量は海域全体では前年比1.2倍の4.9千トンとなった(図7).海域別にみると北部海域は前年比1.3倍の4.2千トンとやや増加したものの,南部海域は前年比0.7倍の0.7千トンとなり減少した.

\ \ \ \ \ \ \ \ \ \ 
体長組成の推移をみると,2018年は2016,2017年と比較して体長20cm台の小型個体が大きく増加した(図8).これらの小型個体は現存尾数が多かった2017年の0歳魚であると考えられ,2019年以降,1歳魚以上の大型個体の増加が期待される.

\subsection{マダラ0歳魚}
マダラ0歳魚は例年水深250m帯を中心に南部海域から北部海域にかけて広く分布が確認されている(図9)。2015年,2016年には水深250m帯に加えて水深350m帯でも分布密度が高かったが,2017年では両水深帯で分布密度は低下した。2018年は北部海域では高密度点は見られなくなり,南部Eラインの250m帯とFラインの350m帯のみでやや高い分布密度を示した。

\ \ \ \ \ \ \ \ \ \ 
海域全体の現存尾数および現存量の推移をみると,2015年,2016年の2年間で資源は増加傾向にあったが,2017年以降減少に転じ,2018年は増加前の2013年と同程度の水準となった(図10)。2018年の現存尾数は海域全体では前年比0.6倍の17百万尾であった。海域別にみると北部海域は前年比0.1倍の1.0百万尾,南部海域は前年比0.7倍の1.6百万尾となり,北部海域における減少が顕著であった。現存量は海域全体では前年比0.4倍の0.4千トンであった。海域別にみると北部海域は前年比0.1倍の0.02千トン,南部海域は前年比0.5倍の0.4千トンとなり,現存尾数同様,北部海域での減少が顕著となった。

\ \ \ \ \ \ \ \ \ \ 
体長組成の推移をみると,例年南部海域の最頻値が北部海域より大きい値を示していたが,2015年,2016年では南部海域で小型化した(図11)。2018年の最頻値は北部海域では10cm,南部海域では11cmであり,南北で同程度となった。東北海域のマダラの満3歳における成熟率は体長に依存している(Narimatsu et al. 2010)。また,満3歳時の体長は0歳時の体長と正の相関があることが知られている(成松 2006)。したがって0歳時の体長はその後の成熟率に影響するとみられ,2015年と2016年にみられたの南部海域の0歳魚の小型化による資源動向への影響を今後注視する必要がある。

\subsection{マダラ1歳魚}
マダラ1歳魚は例年水深250m帯~350m帯を中心に南部海域から北部海域にかけて広く分布が確認されており,2011年3月の東日本大震災(以下震災)以降は,より深い水深550m帯まで分布範囲が広がるとともに,南部海域で分布密度が高い傾向が認められている(図12)。2017年は450m~550m帯の分布密度が特に低かったが,2018年の調査では再び水深450m帯でも比較的分布密度が高い点が認められるようになった。

\ \ \ \ \ \ \ \ \ \ 
マダラ1歳魚の現存尾数及び現存量は2011年に急増したが,その後は減少傾向にある。2018年の現存尾数は前年比1.6倍の18.8百万尾(北部海域6.4百万尾,南部海域12.4百万尾),現存量は前年比1.1倍の3.4千トン(北部海域1.2千トン,南部海域2.1千トン)であり,2016年と同程度であった(図13)。

\subsection{マダラ2歳魚以上}
マダラ2歳魚以上は例年水深250~550m帯を中心に南部海域から北部海域にかけて広く分布している(図15)。また,東日本大震災以降では,2011年は前年とほぼ同様の分布密度であったものの,2012年以降は海域全体で分布密度が高い状況が続いていた。しかし,2017年以降分布密度は低下し,震災前の2010年と類似した分布となっている。2018年の分布は例年分布密度が高いDラインの水深350~450m帯での分布密度が低下していた。また,2017年と比べると,Fライン以南の分布密度も低かった。

\ \ \ \ \ \ \ \ \ \ 
マダラ2歳魚以上の現存尾数および現存量は2012年に急増したが,その後は減少傾向が続いている(図16)。2012年の急増については,2010年級は卓越年級ではなかったにも関わらず,2012年に2歳魚以上が急増したことから,良好な加入による増加というよりも震災の影響による漁獲圧減少によって,生残率が増加したことが原因と考えられている(Narimatsu et al. 2017)。2018年の現存尾数は前年比0.7倍の2.2百万尾(北部海域百万尾,南部海域百万尾),現存量は前年比0.6倍の2.5千トン(北部海域1.4千トン,南部海域1.1千トン)といずれも減少し,急増前の2010年や2011年と同程度の水準となった。

\subsection{イトヒキダラ}
イトヒキダラは例年A~Hラインの水深350m~900帯に広く分布している(図18)。南部海域のG,Hラインでは,分布密度の高い点が確認されている。イトヒキダラの産卵場は東北海域の南部海域~伊豆七島沖にあり,そこで生まれた仔稚魚および小型魚の成育場は南部海域であると考えられている(野別 2002, Hattori et al. 2009)。また,本種の成熟個体は毎年必ずしも産卵を行なうとは限らないとされ(野別 2002),資源は数年に一度発生する卓越年級が支えている。2018年では,産卵場に近い南部海域で高密度分布が見られない一方,北部海域においては10千尾/㎢以上の高密度分布が確認された。

\ \ \ \ \ \ \ \ \ \ 
2018年の現存尾数は海域全体では前年比1.4倍の32.0百万尾であった(図19)。海域別にみると北部海域は前年比2.2倍の25.1百万尾,南部海域では前年比0.6倍の6.9百万尾であった。海域全体の現存量は前年比0.8倍の11.5千トンであった(図19)。海域別にみると北部海域は前年比1.1倍の7.7千トン,南部海域では前年比0.6倍の3.8千トンであった。北部海域では現存尾数は前年から顕著に増加したものの,現存量は昨年と同程度であった。

\ \ \ \ \ \ \ \ \ \ 
体長組成を見ると,北部海域では例年体長30cm以上の個体がほとんどを占める傾向にあったが,2018年の調査では体長30cm以下の小型個体が多く出現した(図20)。2018年調査において,例年南部海域で見られる小型魚の高密度分布が北部海域で出現した理由は明らかではない。この小型魚の分布の北偏が今後も継続するかどうか,注視していく必要がある。なお,本種は底層だけでなく海底から数10mの近底層まで分布する(Yokota and Kawasaki 1990)ことから,着底トロール調査のみからは資源の全容をとらえきれていない可能性があることには注意が必要である。

\subsection{キチジ}
2018年のキチジは例年同様A~Hラインの水深350m~900m帯に幅広く分布していた(図21)。南部海域では2008年以降分布密度が低い傾向が続いており,2018年も南部海域の分布密度は2007年以前と比べて低い状態であった。
現存尾数は2012年以降増加傾向にあったものの,2016年をピークに減少傾向に転じている(図22)。2018年の現存尾数は海域全体で51.3百万尾(前年比0.9倍)となり,増加前の2012年と同程度であった。また,現存量は1995年以来海増加傾向が続き,2014年以降はほぼ横ばいで推移してきたが,2018年の現存量は7.4千トン(前年比0.9倍)となり,減少に転じた(図22)。

\ \ \ \ \ \ \ \ \ \ 
体長組成の推移をみると,1999年~2003年頃には体長10cm未満の小型魚の山が出現している(図23)。1999年以降の現存量増加はこれらの山の成長によるものと考えられる。2014年以降,体長10㎝未満に再び分布の山が認められてきたが,2018年ではそれらの山が不明瞭となっており,今後の動向を注視する必要がある。

\ \ \ \ \ \ \ \ \ \ 
キチジは本調査を開始した1990年代後半に比べて現存尾数,現存量ともに高い水準を維持しているものの,2018年は現存尾数,現存量ともに減少傾向となり,小型魚の山も不明瞭となった。本海域のキチジの再生産成功率は2004年以降低い状態が続いており(森川ほか2019),親魚量は増加しているものの加入量の増加につながっていない。本種の加入量は初期生活期の生残に強く影響されることが示唆されていることから(服部ほか 2006),今後仔稚魚期の生態を調査し,加入量を左右する要因を明らかにする必要があると考えられる。

\subsection{ズワイガニ}
ズワイガニは例年雌雄ともに水深250m~650m帯に分布し,高密度分布はEライン以南で見られる。2007年にはHライン,2015年にはEラインで高密度分布が確認されたものの,2018年にはそのような高密度分布はみられなかった(図24および27)。

\ \ \ \ \ \ \ \ \ \ 
2018年調査では現存尾数は雌で前年比0.7倍の134万尾(北部海域31万尾,南部海域103万尾),雄では217万尾(北部海域41万尾,南部海域176万尾)となり,雌雄ともに過去最低水準を更新した(図25および28)。現存量は雌で前年比0.8倍の126トン(北部海域22トン,南部海域104トン),雄では前年比0.9倍の357トン(北部海域35トン,南部海域322トン)となり,これは雌では調査開始以来2番目に低い水準,雄では5番目に低い水準となった(図25および28)。

\ \ \ \ \ \ \ \ \ \ 
甲幅組成を見ると2018年調査では雌は甲幅6.5~7cm,雄は7~8 cmが中心であるが,雌雄ともに甲幅5cm以下の小型個体が少なかった(図26および29)。小型個体は2011年や2014年に多く出現した例があるが,近年は小型個体の大きな山は出現していない。

\ \ \ \ \ \ \ \ \ \ 
東北海域のズワイガニは大部分が福島県で漁獲されている。震災以降は福島県船による操業は試験操業のみとなっており,漁獲圧が非常に低い状態が続いているにも関わらず,本種の資源は低い水準で推移している。資源が増加しない原因は特定されていないが,震災後は自然死亡係数が増加していることが示唆されている(柴田ほか2018)。自然死亡係数が増加している理由としては,高水温による斃死や分布域の変化,高次捕食者の増加による捕食圧の高まりなどが考えられている(伊藤ほか2014, 柴田ほか2018)が,その実態は明らかになっておらず,今後の研究が望まれる。

\subsection{アカガレイ}
アカガレイは震災以降,現存量,現存尾数ともに減少傾向にあり,2016,2017年は2年連続で過去最低水準となった。2018年はやや回復したものの,現存量では調査開始以来2番目に低い水準,現存尾数では調査開始以来3番目に低い水準となった。

\subsection{サメガレイ}
サメガレイは2016年,2017年には現存尾数,現存量ともに北部海域で高い水準となったが,2018年はそのような高密度点は見られず,北部海域の現存量,現存尾数は大幅に減少した。一方,南部海域では現存尾数は前年と同程度を維持し,現存量は前年比2.6倍に増加した(前年比2.6倍)。

\subsection{ババガレイ}
ババガレイは2006年頃までは現存量,現存尾数ともに北部海域で分布が多く見られていた。震災以降は南部海域で増加し,近年は高水準となっていたが,2018年は南部海域において現存量,現存尾数ともに前年から大きく減少し,震災前と同程度の水準となった。

\ \ \ \ \ \ \ \ \ \ 
%\flushleft{\textbf{\large{参考文献}}}\\
\section{参考文献}
\hangindent=30pt
\noindent
橋本良平, 石戸芳男 (1987) 東北海区のスケトウダラ卵・稚仔の分布. 漁業資源研究会議北日本底魚部会報 20:1-11

\noindent
Hattori T,Narimatsu Y,Nobetsu T,Ito M (2009) Recruitment of threadfin hakeling \textit{Laemonema longipes} off the Pacific coast of northern Honshu, Japan. Fish Sci 75:517-519

\noindent
服部 努, 成松庸二, 伊藤正木, 上田祐司, 北川大二 (2006) 東北海域におけるキチジの
資源量と再生産成功率の経年変化. 日本水産学会誌 72:374 -381

\noindent
伊藤正木, 服部 努, 成松庸二, 柴田泰宙 (2014) 東北沖太平洋におけるマダラによるズ
ワイガニの捕食について. 東北底魚研究 34:123-132

\noindent
北川大二, 服部努 (1998) 調査船による底魚類の資源評価とモニタリング. 水産海洋研究 62:32-36

\noindent
\textcolor{red}{森川英祐, 成松庸二, 柴田泰宙, 鈴木勇人, 時岡 駿, 永尾次郎 (2019) 平成30 (2018) 年度キチジ太平洋北部の資源評価. 我が国周辺海域の漁業資源評価 1232-1263}

\noindent
Narimatsu Y,Shibata Y,Hattori T,Yano T,Nagao J (2017) Effects of a marine-protected area occurred incidentally after the Great East Japan Earthquake on the Pacific cod (\textit{Gadus macrocephalus}) population off northeastern Honshu, Japan. Fish Oceanogr 26(2):181-192

\noindent
Narimatsu Y,Ueda Y,Okuda T,Hattori T,Fujiwara K,Ito M (2010) The effect of temporal changes in life-history traits on reproductive potential in an exploited population of Pacific cod, \textit{Gadus macrocephalus}. ICES J mar sci 67:1659-1666.

\noindent
成松庸二 (2006) マダラの生活史と繁殖生態―繁殖特性の年変化を中心に―. 水産総合
研究センター研究報告(別冊) 4:137 -146

\noindent
野別貴博 (2002) イトヒキダラ \textit{Laemonema longipes} (Schmidt) の生活史および生態に関する
研究. 北海道大学学位論文 145pp

\noindent
大迫正尚, 加賀吉栄, 藤井 浄 (1986) 襟裳以西海域のスケトウダラ卵を経年的に量的比
較を行うために試みた一方法について. 漁業資源研究会議 北日本底魚部会報 19:53-66

\noindent
鈴木勇人, 成松庸二, 柴田泰宙, 森川英祐, 時岡 駿, 永尾次郎 (2019) 平成30 (2018) 年度イトヒキダラ太平洋系群の資源評価.我が国周辺海域の漁業資源評価 1028-1047

\noindent
柴田泰宙, 成松庸二, 鈴木勇人, 森川英祐, 時岡 駿, 永尾次郎 (2019) 平成30 (2018) 
年度ズワイガニ太平洋北部系群の資源評価. 我が国周辺海域の漁業資源評価 493-556

\noindent
Yokota M, Kawasaki T (1990) Population biology of the forked hake, \textit{Laemonema longipes} 
 (Schmidt), off the eastern coast of Honshu, Japan. Tohoku Journal of Agricultural Research 
\textcolor{red}{ページ数は?}



\end{linenumbers}
\end{document}