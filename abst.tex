\documentclass[12pt]{article}
\usepackage{times}
\usepackage{plext}
\usepackage{cite}
\usepackage{setspace}
\usepackage{float}
\usepackage{lineno}
\usepackage{color}
\usepackage[dvipdfmx]{graphicx}
\usepackage{array, booktabs}
\usepackage[top=25truemm,bottom=25truemm,left=30truemm,right=30truemm]{geometry}
\usepackage{threeparttable}
\usepackage{ulem}
%\usepackage{amsmath}
\renewcommand{\figurename}{Fig.}
\renewcommand{\tablename}{Table}
\begin{document}\setcounter{page}{1}
\renewcommand\citeleft{(}
\renewcommand\citeright{)}

\flushleft{\textbf{2020年底魚類現存量調査結果}}\\
%\flushleft{\textbf{Running title: }}\\
\flushright{金森由妃・成松庸二・鈴木勇人・森川英祐・時岡 駿・三澤 遼・永尾次郎(水産資源研究所)
}\\
\ \\

\flushleft{\textbf{背景と目的}\\}
国連海洋法条約では,批准国は領海内の水産資源を適切に管理することが義務付けられている.水産研究・教育機構では,1995年から東北地方太平洋岸沖(東北海域)において底魚類の資源量調査を実施し,主要底魚類の資源状態を調査している.本報告は,2020年秋季に行った調査結果から主要魚種の現存量,分布および体長組成を推定(集計)し,過去の結果と比較することで東北海域における主要底魚類の資源状況を的確に把握することを目的とした.
\ \\

\flushleft{\textbf{材料と方法}\\}
2020年9月27日~11月22日に青森県尻屋崎沖(北緯$\textrm{41}^\circ \textrm{14}^\prime$)から茨城県日立沖(北緯$\textrm{36}^\circ \textrm{29}^\prime$)までの海域で調査船若鷹丸(水産研究・教育機構所属,692トン)を用いた着底トロール調査を実施した.等深線を横切る8本の調査ライン(A~Hライン)を設定し,A~Dラインを北部海域,E~Hラインを南部海域とした.各調査ラインにおいて水深100m~1000mの間に調査点を設定し,合計107地点で調査を実施した.採集された全魚種について尾数と重量を測定し,主要魚種については体長あるいは甲幅測定を行った.スケトウダラは0歳魚と1歳魚以上,マダラは0歳魚,1歳魚および2歳魚以上,ズワイガニは雌雄に区別して測定した.得られたデータから面積-密度法を用いて南北海域別に現存量・現存尾数と体長組成を推定し,過去の結果と比較した.なお,採集効率は1と仮定した.

\flushleft{\textbf{結果と考察}\\}
2020年9月27日~11月22日に青森県尻屋崎沖(北緯$\textrm{41}^\circ \textrm{14}^\prime$)から茨城県日立沖(北緯$\textrm{36}^\circ \textrm{29}^\prime$)までの海域で調査船若鷹丸(水産研究・教育機構所属,692トン)を用いた着底トロール調査を実施した.等深線を横切る8本の調査ライン(A~Hライン)を設定し,A~Dラインを北部海域,E~Hラインを南部海域とした.各調査ラインにおいて水深100m~1000mの間に調査点を設定し,合計107地点で調査を実施した.採集された全魚種について尾数と重量を測定し,主要魚種については体長あるいは甲幅測定を行った.スケトウダラは0歳魚と1歳魚以上,マダラは0歳魚,1歳魚および2歳魚以上,ズワイガニは雌雄に区別して測定した.得られたデータから面積-密度法を用いて南北海域別に現存量・現存尾数と体長組成を推定し,過去の結果と比較した.なお,採集効率は1と仮定した.


\end{document}
